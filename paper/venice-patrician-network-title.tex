\documentclass[preprint,12pt,authoryear]{elsarticle}\usepackage[]{graphicx}\usepackage[]{xcolor}

\begin{document}

\title{Keeping up with the Contarinis: Understanding family ties and power dynamics in the Republic of Venice through social network analysis\tnoteref{t1}}

\author[add1]{J. J. Merelo}
\ead{jmerelo@ugr.es}
\address[add1]{Computer Engineering, Automatics and Robotics department,
  University of Granada (Spain)}
\tnotetext[t1]{This paper has been supported in part by
    project and DemocratAI PID2020-115570GB-C22. Funding for open access charge: Universidad de Granada / CBUA}


\newcommand{\coll}{\emph{colleganza}}
\newcommand{\colln}{\coll{} network}
\newcommand{\serr}{\emph{Serrata}}

\begin{frontmatter}

\begin{abstract}

For the best part of a millennium, the Republic of Venice was a state with a
stable form of government equivalent to an elective monarchy. Rulers, called
\emph{doges} (singular \emph{dux} or \emph{doxe} in the vernacular language),
were chosen from the noble families following a complicated procedure that
essentially guaranteed a very wide majority was needed to support that specific
person. The need for a qualified majority implied that social capital was
essential to achieve that specific job; this was also true for every other job
in the host of institutions that supported the government. Social capital was
accrued by families, not by individuals, since nobility was hereditary and there
were certain restrictions to having several members of the same family in
collegiate institutions; and this was done through commercial, political, and
also social acts: trade and mutual investment or joint ventures, support through
endorsements for government jobs, and also marriage. Thus, understanding the
social network and its evolution is essential to gather some insights on the
tenure and standing of some families, some of which managed to "make doge" for
several centuries, as well as how \emph{new} families had a head start to those
very coveted positions and how others simply vanished. In this paper we will,
through samples of that social network obtained from existing and available open
sources, namely, marriages of doges, registered marriages, and joint commercial
ventures, analyze the main actors in those social networks, as well as study its
dynamics and how it correlates to other events in the history of the
Republic. We will also try and check if there was some hints of strategizing for
better positions in the social networks for the families involved. The position
of certain families in the social network will be analyzed and matched to their
historical record.

\end{abstract}

\begin{keyword}
Social network analysis \sep digital humanities \sep Venice \sep Venetian republic \sep doges \sep history \sep commercial networks
\end{keyword}

\end{frontmatter}

\end{document}
